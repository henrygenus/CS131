\documentclass[9pt]{extarticle}
\usepackage[T1]{fontenc}
\usepackage[textheight=10in]{geometry}
\usepackage[english]{babel}
\usepackage{enumitem}
\usepackage{amsthm} 
\usepackage{amssymb}
\usepackage{amsmath}
\usepackage{mathtools}
\usepackage{graphicx}
\usepackage{wrapfig}
\graphicspath{ {Images/} }
\usepackage[edges]{forest}
\usepackage{tikz-qtree}
\usepackage{float}
\usepackage{multicol}
\usepackage{xcolor}
\usepackage{listings}
\lstset{basicstyle=\ttfamily,
	showstringspaces=false,
	commentstyle=\color{red},
	keywordstyle=\color{blue}
}
\usepackage{tikz}
\usetikzlibrary{positioning, calc, shapes.geometric, shapes.multipart, 
	shapes, arrows.meta, arrows, 
	decorations.markings, external, trees}
\tikzset{
	treenode/.style = 	{shape=rectangle, rounded corners,
					draw, align=center,
					top color=white, bottom color=blue!30},
	no/.style = 	{treenode, bottom color=red!30},
	yes/.style = 	{treenode, bottom color=green!30},
	env/.style = 	{treenode, font=\ttfamily\normalsize},
}
\usepackage{blindtext}
\usepackage{subfiles}
\tolerance=1
\emergencystretch=\maxdimen
\hyphenpenalty=10000
\hbadness=10000


\title{CS131: Programming Languages\\Lecture Notes}
\author{Henry Genus}
\date{Spring 2020}

\begin{document}
\maketitle
\tableofcontents

\newpage
\section{Choice of Notation}
\subfile{lectures/Lecture_1/Choice_of_Notation}

\newpage
\section{Syntax}
\subfile{lectures/Lecture_2/Syntax}

\newpage
\section{Grammar Notation}
\subfile{lectures/Lecture_3/Grammar_Notation}

\newpage
\section{Ambiguity}
\subfile{lectures/Lecture_4/Ambiguity}

\newpage
\section{Software Construction for Programming Languages}
\subfile{lectures/Lecture_5/Software_Construction}

\newpage
\section{Identifiers}
\subfile{lectures/Lecture_6/Identifiers}

\newpage
\section{Polymorphism}
\subfile{lectures/Lecture_7/Polymorphism}

\newpage
\section{Scope and Error Handling}
\subfile{lectures/Lecture_8/Scope_and_Error_Handling}

\newpage
\section{Memory Management}
\subfile{lectures/Lecture_9/Memory_Management}

\newpage
\section{Object-Oriented Programming}
\subfile{lectures/Lecture_10/OOP}

\appendix
 
\newpage
\section{OCaml}
\subfile{lectures/Languages/OCaml}

\newpage
\section{Java}
\subfile{lectures/Languages/Java}

\newpage
\section{Prolog}
\subfile{lectures/Languages/Prolog}

\newpage
\section{Scheme}
\subfile{lectures/Languages/Scheme}

\end{document}